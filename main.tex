\documentclass[10pt,a4paper]{article}
\usepackage[utf8]{inputenc}
\usepackage{longtable}
\usepackage{graphicx}
\usepackage{array}
\usepackage{amsmath}
\usepackage{enumitem}
\usepackage{caption}
\usepackage{wrapfig}
\usepackage{subfig}
\usepackage{floatrow}
\usepackage{tikz}
\usepackage{float}
\renewcommand{\baselinestretch}{1.5}
\usetikzlibrary{shapes,arrows}
\usetikzlibrary{positioning}
\usetikzlibrary{shapes,arrows,arrows,positioning,fit}
\usepackage{verbatim}
\usepackage[a4paper,left=3cm,right=1.5cm,top=3cm,bottom=3cm,headheight=12.5pt]{geometry}
\usepackage{scrextend}
\changefontsizes{13pt}
\usepackage{lipsum}
\usepackage{fancyhdr}
\usepackage{natbib}
\usepackage{multicol}
\title{Review about Drug deliver nanosystem}
\author{Tuan Thanh Nguyen}
\date{December 2018}

\begin{document}
\maketitle
\begin{abstract}
    Magnetite nanoparticles (MNPs), Doxorubicin (Dox) and Curcumin (Cur) were known as remarkable anticancer agents. However, To make the treatments act effectively, drug deliver have important role to improve properties of these medicines. Along with the development of cancer treatment, drug delivery nanosystems (DDNSs) have attracted a great deal of concern among scientists over the world. DDNSs not only improve water solubility of anticancer drugs but also increase therapeutic efficacy and minimize the side effects of treatment methods through targeting mechanisms including passive and active targeting\cite{bhbh1}. In this review, I present some properties of chemotherapy mentioned and introduce some polymer system used to deliver those.
\end{abstract}

\begin{multicols}{2}
\section{Introduction}

Nanotechnology had many outstanding development in recent time. This development create many keys for solving scientific problem. Drug deliver is one of main mission that nanotechnology intake of. Especially, cancer treatments require devices carrying high accuracy to tumors.

Doxorubicin, curcumin, MNPs are compounds behaving potential in chemotherapeutic agents. While Dox, MNPs were proposed many mechanisms in anti-cancer processes but Cur still be studied and have not had clear theoretical explanation for confirming that Cur is a amazing anti-cancer drug. Nevertheless, Curcumin still be a interesting compound for medicine like antioxidant, antimicrobial, anti-inflammatory and anti-carcinogenic.

Despite having many advantages to become therapeutic drug, both of them have to deal with issue in solubility. Familiar solutions are incorporation of this drug into liposomes and into lipid vesicles, co-precipitation and surface modified. There are variety of polymers have been studied to adapt the demand of those methods. In this review, the author would like to focus on some polymers inc+lude: 1,3-$\beta$ Glucan, O-carboxymethyl chitosan(OCMCs), PLA-TPLS copolymer. Which have excellent properties in bioavaibility, biodegradability and solubility.

Moreover, In this review, author will have a description for techniques are used to analysis, determine properties and efficiency of those compound. There include FTIR, FE-SEM, UV-Vis, XRD and H-NMR.

\section{Theurapeutic agent and mechanism}
\subsection{Doxorubicin(Dox)}

Doxorubicin is a chemotherapy medication used to treat cancer.  
\subsection{Fe$_3$O$_4$ nanoparticle(MNPs)}

There are three main properties mined for application. Firstly, the unique ability of MNPs to be guided by an external magnetic field has been utilized for targeted drug and gene delivery,tissue engineering, cell tracking and bioseparation. 
econdly,with the ability to perturb magnetic local field, they can serve as effective contrast enhancer in magnetic resonance imaging(MRI). Finally, MNPs can effectively adsorb energy from external alternating magnetic field to create a nanosized heating source that is used as thermoseedin magnetic inductive heating(MIH) hyperthermia. The combination of the first with  the second and/or the third application makes MNPs, in fact, a multifunctional nanovector\cite{2}. 

MNPs destroy cancer cell during magnetic nanoparticle hyperthermia which is cancer therapy employing heat dissipation by magnetic nanoparticles in an alternating magnetic field to kill tumor cells. 


\bibliographystyle{plain}
\bibliography{reference.bib}
\end{multicols}

\end{document}
